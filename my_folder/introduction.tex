\chapter*{Введение} % * не проставляет номер
\addcontentsline{toc}{chapter}{Введение} % вносим в содержание
\par Флуоресцентная микроскопия - это метод получения снимков светящихся объектов малых размеров. Он широко используем в разных областях: от материаловедения до нейробиологии. Флуоресцентная микроскопия имеет ряд преимуществ перед другими формами микроскопии, предлагая высокую чувствительность и специфичность. \cite{ROST2017627}
\par Однако данный метод обладает и своими недостатками. Один из них заключается в том, что исходные изображения, получаемые с микроскопа, содержат шумы и искажения, которые могут существенно затруднить последующий анализ объекта. Поэтому учёным и экспертам требуются предварительный ряд обработок над изображениями (деконволюция, денойзинг). Кроме того, существуют ручные процессы, которые занимают много времени и необходимы для обработки изображений, поэтому требуют автоматизации. Один из таких процессов — это сегментация калибровочных мелких светящихся объектов (например, флуоресцентных сфер). В последние годы наблюдается значительный прогресс в области флуоресцентной микроскопии, что позволяет ученым получать изображения с все более высокой разрешающей способностью.  
\par Целью работы является разработка ПО в виде онлайн-сервиса, способного улучшать изображения биологических образцов с помощью различных методов, включая аналитические и нейросетевые алгоритмы. Задачи, которые решает данный сервис, включают автоматическую сегментацию флуоресцентных сфер, алгоритм денойзинга (обесшумливания) трёхмерных изображений, комбинирующий в себе аналитический подход и метод глубокого обучения, а также деконволюцию (улучшение качества) изображений. Результат работы позиционируется как инструмент, который поможет ученым и исследователям получать более точные изображения клеточных структур и материалов, что позволит им делать более качественные выводы.



%% Вспомогательные команды - Additional commands
%\newpage % принудительное начало с новой страницы, использовать только в конце раздела
%\clearpage % осуществляется пакетом <<placeins>> в пределах секций
%\newpage\leavevmode\thispagestyle{empty}\newpage % 100 % начало новой строки