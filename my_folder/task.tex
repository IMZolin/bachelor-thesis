%%%% Начало оформления заголовка - оставить без изменений !!! %%%%
\input{my_folder/task_settings}	% настройки - начало 
	
				{%\normalfont %2020
						\MakeUppercase{\SPbPU}}\\
				\institute

\par}\intervalS% завершает input

				\noindent
				\begin{minipage}{\linewidth}
				\vspace{\mfloatsep} % интервал 	
				\begin{tabularx}{\linewidth}{Xl}
					&УТВЕРЖДАЮ      \\
					&\HeadTitle     \\			
					&\underline{\hspace*{0.1\textheight}} \Head     \\
					&<<\underline{\hspace*{0.05\textheight}}>> \underline{\hspace*{0.1\textheight}} \thesisYear г.  \\  
				\end{tabularx}
				\vspace{\mfloatsep} % интервал 	
				\end{minipage}

\intervalS{\centering\bfseries%

				ЗАДАНИЕ\\
				на выполнение %с 2020 года 
				%по выполнению % до 2020 года
				выпускной квалификационной работы


\intervalS\normalfont%

				студенту {\AuthorFullDat{} гр.~\group}


\par}\intervalS%
%%%%
%%%% Конец оформления заголовка  %%%%
 	
	
	
\begin{enumerate}[1.]
	\item Тема работы: {\expandafter \thesisTitle.}
	%\item Тема работы (на английском языке): \uline{\thesisTitleEn.} % вероятно после 2021 года
	\item Срок сдачи студентом законченной работы: \thesisDeadline 
	\item Исходные данные по работе:\\ Изображения различных флуоресцентных объектов: сферы, нейроны, трубочки, клетки; параметры съёмки. Данные для экспериментов были получены с использованием конфокальных микроскопов биологических объектов различного типа при сотрудничестве с Лабораторией молекулярной нейродегенерации.\\
	Инструментальные средства:
	\begin{itemize}
		\item Языки программирования Python, JavaScript 
		\item Среда разработки PyCharm
		\item Программная библиотека компьютерного зрения OpenCV
		\item Программная библиотека глубокого обучения PyTorch
		\item Система контроля версий git
	\end{itemize}
	%
	%\printbibliographyTask % печать списка источников % КОММЕНТИРУЕМ ЕСЛИ НЕ ИСПОЛЬЗУЕТСЯ
	% В СЛУЧАЕ, ЕСЛИ НЕ ИСПОЛЬЗУЕТСЯ МОЖНО ТАКЖЕ ЗАЙТИ В setup.tex и закомментировать \vspace{-0.28\curtextsize}
	\item Содержание работы (перечень подлежащих разработке вопросов):
	\begin{enumerate}[label=\theenumi\arabic*.]
		\item Введение. Обоснование актуальности
		\item Постановка задачи
		\item Обзор существующих решений
		\item Обзор алгоритма автоматической сегментации флуоресцентных сфер
		\item Обзор метода глубокого обучения шумоподавления
		\item Результаты обучения
		\item Тестирования метода шумоподавления
		\item Обзор онлайн-сервиса
		\item Заключение
	\end{enumerate}
		\item Дата выдачи задания: \thesisStartDate.
\end{enumerate}

\intervalS%можно удалить пробел

Руководитель ВКР \uline{\hspace*{0.1\textheight}}\Supervisor


\intervalS%можно удалить пробел

Консультант \uline{\hspace*{0.1\textheight}}\ConsultantExtra


\intervalS%можно удалить пробел

%Консультант по нормоконтролю \uline{\hspace*{0.1\textheight} \ConsultantNorm}%ПОКА НЕ ТРЕБУЕТСЯ, Т.К. ОН У ВСЕХ ПО УМОЛЧАНИЮ

Задание принял к исполнению \thesisStartDate

\intervalS%можно удалить пробел

Студент \uline{\hspace*{0.1\textheight}}\Author



\input{my_folder/task_settings_restore}	% настройки - конец