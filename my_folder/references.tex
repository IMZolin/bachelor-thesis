%%% Не мянять - Do not modify
%%
%%
\clearpage                                  % В том числе гарантирует, что список литературы в оглавлении будет с правильным номером страницы
%\hypersetup{ urlcolor=black }               % Ссылки делаем чёрными
%\providecommand*{\BibDash}{}                % В стилях ugost2008 отключаем использование тире как разделителя 
\urlstyle{rm}                               % ссылки URL обычным шрифтом
\ifdefmacro{\microtypesetup}{\microtypesetup{protrusion=false}}{} % не рекомендуется применять пакет микротипографики к автоматически генерируемому списку литературы
%\newcommand{\fullbibtitle}{Список литературы} % (ГОСТ Р 7.0.11-2011, 4)
%\insertbibliofull  
%\noindent
%\begin{group}
\chapter*{Список использованных источников}	
\label{references}
\addcontentsline{toc}{chapter}{Список использованных источников}	% в оглавление 
\printbibliography[env=SSTfirst]  
\begin{enumerate}[]
	\item Sibarita, Jean-Baptiste. (2005). Deconvolution Microscopy. Advances in biochemical engineering/biotechnology. 95. 201-43. 10.1007/b102215.
	\item Yide Zhang, Yinhao Zhu, Evan Nichols, Qingfei Wang, Siyuan Zhang, Cody Smith, Scott Howard. (2015) «A Poisson-Gaussian Denoising Dataset with Real Fluorescence Microscopy Images», 2019 https://arxiv.org/pdf/1812.10366.pdf 
	\item Gayathri and A. Srinivasan, "An efficient algorithm for image denoising using NLM and DBUTM estimation," TENCON 2014 - 2014 IEEE Region 10 Conference, Bang-kok, Thailand, 2014, pp. 1-6, doi: 10.1109/TENCON.2014.7022388. \url{https://www.researchgate.net/publication/282987660_An_efficient_algorithm_for_image_denoising_using_NLM_and_DBUTM_estimation}
	\item GoodfellowI.J., Bengio Y., CourvilleA. Deep Learning.—Cambridge,MA,
	USA:MITPress,2016.—(http://www.deeplearningbook.org).
\end{enumerate}   


                    % Подключаем Bib-базы
%\ifdefmacro{\microtypesetup}{\microtypesetup{protrusion=true}}{}
%\urlstyle{tt}                               % возвращаем установки шрифта ссылок URL
%\hypersetup{ urlcolor={urlcolor} }          % Восстанавливаем цвет ссылок



%\urlstyle{rm}                               % ссылки URL обычным шрифтом
%\ifdefmacro{\microtypesetup}{\microtypesetup{protrusion=false}}{} % не рекомендуется применять пакет микротипографики к автоматически генерируемому списку литературы
%\insertbibliofull                           % Подключаем Bib-базы
%\ifdefmacro{\microtypesetup}{\microtypesetup{protrusion=true}}{}
%\urlstyle{tt}                               % возвращаем установки шрифта ссылок URL
