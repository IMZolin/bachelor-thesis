\chapter*{Заключение} \label{ch-conclusion}
\par В данной работе были рассмотрены существующие методы обесшумливания двухмерных изображений, такие как нелокальный фильтр среднего (Non-local means), а также метод глубокого обучения Noise2Noise. На основе изученного материала был разработан алгоритм 3D денойзинга, включающий в себя эти 2 подхода, а также комбинирующий их. На основе полученных результатов денойзинга можно сделать следующие выводы:
\begin{itemize}[]
	\item Нелокальный фильтр демонстрирует высокую скорость работы и эффективно удаляет шумы с низкой интенсивностью.
	\item Нейросетевой метод позволяет удалять шумы на всём изображении. Метрики качества, такие как PSNR и RMSE, превосходят показатели нелокального фильтра примерно  на 15\%.
	\item Комбинирование вышеупомянутых методов позволило добиться улучшения метрик PSNR и RMSE на 10\% по сравнению с нейросетевым подходом, а также обеспечивают наиболее гладкий график интенсивностей на выходе.
	\item Шум, наблюдаемый на реальных снимках приближен к распредению Пуассона.
	\item Результы методов на реальных снимках показывают уменьшение дисперсии шума, что также доказывает работоспобность методов шумоподавления.
	\item Комбинированные методы имеют меньшие значения средних и дисперсий, что также подтверждает улучшения качества работы при комбинации методов.
\end{itemize}
\par Также был разработан алгоритм автоматической сегментации флуоресцентных сфер, который позволил ускорить процесс ручной сегментации на 2 порядка.
\par Все разработнные алгоритмы были успешно интегрированы и развернуты в веб-сервисе, созданном в сотрудничестве с Лабораторией молекулярной нейродегенерации. 