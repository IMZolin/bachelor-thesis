%%% Внесите свои данные - Input your data
%%
%%
\newcommand{\Author}{И.М.\,Золин} % И.О. Фамилия автора 
\newcommand{\AuthorFull}{Золин Иван Максимович} % Фамилия Имя Отчество автора
\newcommand{\AuthorFullDat}{Золину Ивану Максимовичу} % Фамилия Имя Отчество автора в дательном падеже (Кому? Студенту...)
\newcommand{\AuthorFullVin}{Золина Ивана Максимовича} % в винительном падеже (Кого? что?  Програмиста ...)
\newcommand{\AuthorPhone}{+7-906-936-33-00} % номер телефорна автора для оперативной связи  
\newcommand{\Supervisor}{В.С.\,Чуканов} % И. О. Фамилия научного руководителя
\newcommand{\SupervisorFull}{Чуканов Вячеслав Сергеевич} % Фамилия Имя Отчество научного руководителя
\newcommand{\SupervisorVin}{В.С.\,Чуканова} % И. О. Фамилия научного руководителя  в винительном падеже (Кого? что? Руководителя ...)
\newcommand{\SupervisorJob}{старший преподаватель ВШПМиВФ, научный сотрудник ЛМН} %
\newcommand{\SupervisorJobVin}{} % в винительном падеже (Кого? что?  Програмиста ...)
\newcommand{\SupervisorDegree}{} %
\newcommand{\SupervisorTitle}{} % 
%%
%%
%Руководитель, утверждающий задание
\newcommand{\Head}{К.Н.\,Козлов} % И. О. Фамилия руководителя подразделения (руководителя ОП)
\newcommand{\HeadDegree}{Руководитель ОП}% Только должность:   
%Руководитель %ОП 
%Заведующий % кафедрой
%Директор % Высшей школы
%Зам. директора
\newcommand{\HeadDep}{} % заменить на краткую аббревиатуру подразделения или оставить пустым, если утверждает руководитель ОП

%%% Руководитель, принимающий заявление
\newcommand{\HeadAp}{И.О.\,Фамилия} % И. О. Фамилия руководителя подразделения (руководителя ОП)
\newcommand{\HeadApDegree}{Руководитель ОП}% Только должность:   
%Руководитель ОП 
%Заведующий кафедрой
%Директор Высшей школы
\newcommand{\HeadApDep}{} % заменить на краткую аббревиатуру подразделения или оставить пустым, если утверждает руководитель ОП
%%% Консультант по нормоконтролю
\newcommand{\ConsultantNorm}{Л.А. Арефьева} % И. О. Фамилия консультанта по нормоконтролю. ТОЛЬКО из числа ППС!
\newcommand{\ConsultantNormDegree}{} %   
%%% Первый консультант
\newcommand{\ConsultantExtraFull}{Пчицкая Екатерина Игоревна} % Фамилия Имя Отчетство дополнительного консультанта 
\newcommand{\ConsultantExtra}{Е.И.\,Пчицкая} % И. О. Фамилия дополнительного консультанта 
\newcommand{\ConsultantExtraDegree}{доцент ВШБСиТ, научный сотрудник ЛМН, \\ кандидат физико-математических наук} % 
\newcommand{\ConsultantExtraVin}{И.О.\,Фамилию} % И. О. Фамилия дополнительного консультанта в винительном падеже (Кого? что? Руководителя ...)
\newcommand{\ConsultantExtraDegreeVin}{должность, степень} %  в винительном падеже (Кого? что? Руководителя ...)
%%% Второй консультант
\newcommand{\ConsultantExtraTwoFull}{Фамилия Имя Отчетство} % Фамилия Имя Отчетство дополнительного консультанта 
\newcommand{\ConsultantExtraTwo}{И.О.\,Фамилия} % И. О. Фамилия дополнительного консультанта 
\newcommand{\ConsultantExtraTwoDegree}{должность, степень} % 
\newcommand{\ConsultantExtraTwoVin}{И.О.\,Фамилию} % И. О. Фамилия дополнительного консультанта в винительном падеже (Кого? что? Руководителя ...)
\newcommand{\ConsultantExtraTwoDegreeVin}{должность, степень} %  в винительном падеже (Кого? что? Руководителя ...)
\newcommand{\Reviewer}{И.О.\,Фамилия} % И. О. Фамилия резензента. Обязателен только для магистров.
\newcommand{\ReviewerDegree}{должность, степень} % 
%%
%%
\renewcommand{\thesisTitle}{Онлайн-сервис ИИ-деконволюции и денойзинга для конфокальных микроскопов}
\newcommand{\thesisDegree}{работа бакалавра}% дипломный проект, дипломная работа, магистерская диссертация %c 2020
\newcommand{\thesisTitleEn}{Online AI deconvolution and denoising service for confocal microscopes} %2020
\newcommand{\thesisDeadline}{25.05.2024}
\newcommand{\thesisStartDate}{10.01.2024}
\newcommand{\thesisYear}{2024}
%%
%%
\newcommand{\group}{5030102/00201} % заменить вместо N номер группы
\newcommand{\thesisSpecialtyCode}{01.03.02}% код направления подготовки
\newcommand{\thesisSpecialtyTitle}{Прикладная метематика и информатика} % наименование направления/специальности
\newcommand{\thesisOPPostfix}{02} % последние цифры кода образовательной программы (после <<_>>)
\newcommand{\thesisOPTitle}{Системное программирование}% наименование образовательной программы
%%
%%
\newcommand{\institute}{
Физико-механический институт
%Институт компьютерных наук и~технологий
%Гуманитарный институт
%Инженерно-строительный институт
%Институт биомедицинских систем и технологий
%Институт металлургии, машиностроения и транспорта
%Институт передовых производственных технологий
%Институт прикладной математики и механики
%Институт физики, нанотехнологий и телекоммуникаций
%Институт физической культуры, спорта и туризма
%Институт энергетики и транспортных систем
%Институт промышленного менеджмента, экономики и торговли
}%
%%
%%




%%% Задание ключевых слов и аннотации
%%
%%
%% Ключевых слов от 3 до 5 слов или словосочетаний в именительном падеже именительном падеже множественного числа (или в единственном числе, если нет другой формы) по правилам русского языка!!!
%%
%%
\newcommand{\keywordsRu}{Флуоресцентная микроскопия, Конфокальная микроскопия, Онлайн-сервис,  Компьютерное зрение, Денойзинг, Сегментация, Свёрточные нейронные сети, Архитектура сети} % ВВЕДИТЕ ключевые слова по-русски
%%
%%
\newcommand{\keywordsEn}{Fluorescence microscopy, Confocal microscopy, Online service, Computer vision, Denoising, Segmentation, Convolutional neural networks, Network architecture} % ВВЕДИТЕ ключевые слова по-английски
%%
%%
%% Реферат ОТ 1000 ДО 1500 знаков на русский или английский текст
%%
%Реферат должен содержать:
%- предмет, тему, цель ВКР;
%- метод или методологию проведения ВКР:
%- результаты ВКР:
%- область применения результатов ВКР;
%- выводы.

\newcommand{\abstractRuStart}{Данная работа посвящена разработке онлайн-сервиса, производящего различные процессы улучшения качества трёхмерных изображений люминисцентных объектов малого размера, регистрируемых конфокальными микроскопами. В рамках данного исследования были сформулированы следующие задачи:} % ВВЕДИТЕ текст аннотации по-русски
\newcommand{\abstractRuList}{ 
	\begin{itemize}[]\item Изучение существующих алгоритмов, решающих схожие задачи. \item Разработка алгоритма машинного обучения для автоматической сегментации флуоресцентных сфер. \item Разработка алгоритма глубокого обучения денойзинга (шумоподавления) для трёхмерных изображений. \item Создание онлайн-сервиса, обеспечивающего взаимодействие с реализованными алгоритмами. \end{itemize}}
\newcommand{\abstractRuEnd}{\par В результате проделанной работы был разработан онлайн-сервис, включающий в себя следующий функционал:
	\begin{itemize}[] \item Ручной и автоматический подходы к сегментации флуоресцентных сфер. Автоматический подход базировался на методах компьютерного зрения.
	\item Алгоритм глубокого обучения для денойзинга трёхмерных изображений, основанный на сети Noise2Noise (N2N) и фильтре Non-local means (NLM), обучен и протестирован на данных из разных микроскопов. 
	\item Реализованный метод деконволюции Ричарсдона-Люси для генерация эксперментальной функции рассеяния точки (эФРТ) и улучшения трёхмерных изображений.
	\end{itemize}
\par По итогам работы сделан вывод, что онлайн-сервис эффективен для получения более точных изображений клеточных структур и материалов, что позволяет детально анализировать наблюдаемые объекты в медико-биологических исследованиях.
} % ВВЕДИТЕ текст аннотации
%%
\newcommand{\abstractEnStart}{This work is devoted to the development of an online service that produces various processes to improve the quality of three-dimensional images of small-sized luminescent objects recorded by confocal microscopes. The following objectives were formulated as part of this research:} % ВВЕДИТЕ текст аннотации по-английски
\newcommand{\abstractEnList}{\begin{itemize}[]\item Study of existing algorithms that solve similar problems. \item Development of a machine learning algorithm for automatic segmentation of fluorescent spheres. \item Development of a deep learning denoising (noise reduction) algorithm for three-dimensional images. \item Creation of an online service that provides interaction with the implemented algorithms. \end{itemize}}
\newcommand{\abstractEnEnd}{As a result of this work, an online service was developed that includes the following functionality:
	\begin{itemize}[]\item Manual and automatic approaches to fluorescent sphere segmentation. The automatic approach was based on computer vision methods.
		\item  A deep learning algorithm for denoising 3D images based on Noise2Noise (N2N) network and Non-local means (NLM) filter is trained and tested on data from different microscopes. 
		\item Implemented Richardsdon-Lucy deconvolution method to generate expert point scattering function (ePRF) and improve 3D images.
\end{itemize}
 It is concluded that the online service is effective in obtaining more accurate images of cellular structures and materials, which allows detailed analysis of observed objects in biomedical research.
}

%%% РАЗДЕЛ ДЛЯ ОФОРМЛЕНИЯ ПРАКТИКИ
%Место прохождения практики
\newcommand{\PracticeType}{Отчет о прохождении преддипломной практики}

\newcommand{\Workplace}{СПбПУ, ФизМех, ВШПМиВФ} % TODO Rename this variable

% Даты начала/окончания
\newcommand{\PracticeStartDate}{%
03.05.2024%
%	22.06.2020
}%
\newcommand{\PracticeEndDate}{%
	31.05.2024%
%	18.07.2020%
}%
%%

\newcommand{\School}{
	Высшая школа прикладной математики и вычислительной
	Физики
	
%	Высшая школа интеллектуальных систем и~суперкомпьютерных~технологий 
}
\newcommand{\practiceTitle}{Онлайн-сервис ИИ-деконволюции и денойзинга для конфокальных микроскопов}


%% ВНИМАНИЕ! Необходимо либо заменить текст аннотации (ключевых слов) на русском и английском, либо удалить там весь текст, иначе в свойства pdf-отчета по практике пойдет шаблонный текст.

%%% Не меняем дальнейшую часть - Do not modify the rest part
%%
%%
%%
%%
\ifnumequal{\value{docType}}{1}{% Если ВКР, то...
	\newcommand{\DocType}{Выпускная квалификационная работа}
	\newcommand{\pdfDocType}{\DocType~(\thesisDegree)} %задаём метаданные pdf файла
	\newcommand{\pdfTitle}{\thesisTitle}
}{% Иначе 
	\newcommand{\DocType}{\PracticeType}
	\newcommand{\pdfDocType}{\DocType} %задаём метаданные pdf файла
	\newcommand{\pdfTitle}{\practiceTitle}
}%
\newcommand{\HeadTitle}{\HeadDegree~\HeadDep}
\newcommand{\HeadApTitle}{\HeadApDegree~\HeadApDep}
\newcommand{\thesisOPCode}{\thesisSpecialtyCode\_\thesisOPPostfix}% код образовательной программы
\newcommand{\thesisSpecialtyCodeAndTitle}{\thesisSpecialtyCode~\thesisSpecialtyTitle}% Код и наименование направления/специальности
\newcommand{\thesisOPCodeAndTitle}{\thesisOPCode~\thesisOPTitle} % код и наименование образовательной программы
%%
%%
\hypersetup{%часть болка hypesetup в style
		pdftitle={\pdfTitle},    % Заголовок pdf-файла
		pdfauthor={\AuthorFull},    % Автор
		pdfsubject={\pdfDocType. Шифр и наименование направления подготовки: \thesisSpecialtyCodeAndTitle. \abstractRuStart},      % Тема
		pdfcreator={LaTeX, SPbPU-student-thesis-template},     % Приложение-создатель
%		pdfproducer={},  % Производитель, Производитель PDF % будет выставлена автоматически
		pdfkeywords={\keywordsRu}}
%%
%%
%% вспомогательные команды
\newcommand{\firef}[1]{рис.\ref{#1}} %figure reference
\newcommand{\taref}[1]{табл.\ref{#1}}	%table reference
%%
%%
%% Архивный вариант задания ключевых слов, аннотации и благодарностей 
% Too hard to export data from the environment to pdf-info
% https://tex.stackexchange.com/questions/184503/collecting-contents-of-environment-and-store-them-for-later-retrieval
%заменить NewEnviron на newenvironment для распознавания команды в TexStudio
%\NewEnviron{keywordsRu}{\noindent\MakeUppercase{\BODY}}
%\NewEnviron{keywordsEn}{\noindent\MakeUppercase{\BODY}}
%\newenvironment{abstractRu}{}{}
%\newenvironment{abstractEn}{}{}
%\newenvironment{acknowledgementsRu}{\par{\normalfont \acknowledgements.}}{}
%\newenvironment{acknowledgementsEn}{\par{\normalfont \acknowledgementsENG.}}{}


%%% Переопределение именований %%% Не меняем - Do not modify
%\newcommand{\Ministry}{Минобрнауки России} 
\newcommand{\Ministry}{Министерство науки и высшего образования Российской~Федерации} %с 2020
\newcommand{\SPbPU}{Санкт-Петербургский политехнический университет Петра~Великого}
\newcommand{\SPbPUOfficialPrefix}{Федеральное государственное автономное образовательное учреждение высшего образования}
\newcommand{\SPbPUOfficialShort}{ФГАОУ~ВО~<<СПбПУ>>}
%% Пробел между И. О. не допускается.
\renewcommand{\alsoname}{см. также}
\renewcommand{\seename}{см.}
\renewcommand{\headtoname}{вх.}
\renewcommand{\ccname}{исх.}
\renewcommand{\enclname}{вкл.}
\renewcommand{\pagename}{Pages}
\renewcommand{\partname}{Часть}
\renewcommand{\abstractname}{\textbf{Аннотация}}
\newcommand{\abstractnameENG}{\textbf{Annotation}}
\newcommand{\keywords}{\textbf{Ключевые слова}}
\newcommand{\keywordsENG}{\textbf{Keywords}}
\newcommand{\acknowledgements}{\textbf{Благодарности}}
\newcommand{\acknowledgementsENG}{\textbf{Acknowledgements}}
\renewcommand{\contentsname}{Content} % 
%\renewcommand{\contentsname}{Содержание} % (ГОСТ Р 7.0.11-2011, 4)
%\renewcommand{\contentsname}{Оглавление} % (ГОСТ Р 7.0.11-2011, 4)
\renewcommand{\figurename}{Рис.} % Стиль СПбПУ
%\renewcommand{\figurename}{Рисунок} % (ГОСТ Р 7.0.11-2011, 5.3.9)
\renewcommand{\tablename}{Таблица} % (ГОСТ Р 7.0.11-2011, 5.3.10)
%\renewcommand{\indexname}{Предметный указатель}
\renewcommand{\listfigurename}{Список рисунков}
\renewcommand{\listtablename}{Список таблиц}
\renewcommand{\refname}{\fullbibtitle}
\renewcommand{\bibname}{\fullbibtitle}

\newcommand{\chapterEnTitle}{Сhapter title} % <- input the English title here (only once!) 
\newcommand{\chapterRuTitle}{Название главы}          % <- введите 
\newcommand{\sectionEnTitle}{Section title} %<- input subparagraph title in english
\newcommand{\sectionRuTitle}{Название подраздела} % <- введите название подраздела по-русски
\newcommand{\subsectionEnTitle}{Subsection title} % - input subsection title in english
\newcommand{\subsectionRuTitle}{Название параграфа} % <- введите название параграфа по-русски
\newcommand{\subsubsectionEnTitle}{Subsubsection title} % <- input subparagraph title in english
\newcommand{\subsubsectionRuTitle}{Название подпараграфа} % <- введите название подпараграфа по-русски